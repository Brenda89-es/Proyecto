\documentclass[a4paper,twocolumn,10pt]{article}
\usepackage[utf8x]{inputenc}
\usepackage[spanish]{babel}
\usepackage{amsmath,amsfonts,amssymb}
\usepackage{graphicx}
\usepackage{flushend}
\usepackage{multicol}
\usepackage{multirow}
\author{Brenda Romero Salcedo}
\title{Proyecto final. Review Científico sobre VIH}
\date{14 de Octubre de 2018}
\begin{document}
\twocolumn[
\begin{@twocolumnfalse}
\vspace*{-3cm}
\maketitle
\vspace*{-1cm}
\begin{center}\rule{0.9\textwidth}{0.1mm}\end{center}
\begin{abstract}
\normalsize Un artículo suele empezarse con un resumen. Dicho resumen debe ser claro y conciso, y no tiene que tener referencias bibliográficas. En inglés, abstract significa resumen, y resume significa reanudar. Cuidado no confundas esas dos palabras.\\ \\
{\bfseries Palabras clave:} Manzana, Serpiente.
\begin{center}\rule{0.9\textwidth}{0.1mm}\end{center}
\vspace*{0.5cm}
\end{abstract}
\end{@twocolumnfalse}
]
\section{Introducción}
La melatonina o N-acetil-5-metoxitriptamina es una hormona encontrada en seres humanos, animales, plantas, hongos y bacterias, así como en algunas algas; en concentraciones que varían de acuerdo al ciclo diurno/nocturno. La melatonina es sintetizada a partir del aminoácido esencial triptófano. Se produce, principalmente, en la glándula pineal, y participa en una gran variedad de procesos celulares, neuroendocrinos y neurofisiológicos, como controlar el ciclo diario del sueño.
%%%%%%%%%Introducir una imagen que ocupe ambas columnas%%%%%%%%%%%%%%%%%%%
%\begin{figure*}[htb]
%\centering
%\includegraphics[width=1\textwidth]{./montblanc}
%\caption{Mont Blanc.}
%\label{fig:mont}
%\end{figure*}
%%%%%%%%%%%%Introducir una fórmula matemática%%%%%%%%%%%%%%%%%%%%%%%%%%
%La ecuación de una recta en el plano cartesiano es de la forma
%\begin{equation*}
%ax+by+c=0
%\end{equation*}
%donde $a$, $b$, $c$ son constantes.
\section{Métodos}
Una de las características más sobresaliente respecto a la biosíntesis pineal de melatonina es su variabilidad a lo largo del ciclo de 24 horas, y su respuesta precisa a cambios en la iluminación ambiental. Por ello, la melatonina se considera una neurohormona producida por los pinealocitos en la glándula pineal (localizada en el diencéfalo), la cual produce la hormona bajo la influencia del núcleo supraquiasmático del hipotálamo, que recibe información de la retina acerca de los patrones diarios de luz y oscuridad.
\section{Discusión}
La glándula pineal de los humanos tiene un peso cercano a los 150 miligramos y ocupa la depresión entre el colículo superior y la parte posterior del cuerpo calloso. A pesar de la existencia de conexiones entre la glándula pineal y el cerebro, aquella se encuentra fuera de la barrera hematoencefálica; y está inervada principalmente por los nervios simpáticos que proceden de los ganglios cervicales superiores.
En el Homo sapiens se produce una síntesis constante de melatonina que disminuye abruptamente hacia los 30 años de edad, a partir de la adolescencia suelen producirse calcificaciones en la glándula que reciben el nombre de corpora arenacea.1​ Se ha observado que la melatonina tiene, entre otras funciones, regular el reloj biológico y disminuir la oxidación. Los déficits de melatonina pueden ir acompañados de insomnio y depresión y podrían provocar una paulatina aceleración del envejecimiento.
%%%%%%%%%%%%Introducir una tabla que ocupe una sola columna
%\begin{table}[htb]
%\centering
%\begin{tabular}{|l|l|l|l|}
%\hline
%& \multicolumn{3}{c|}{Europa} \\
%\cline{2-4}
%& Ciudad & Río & Símbolo\\
%\hline \hline
%\multirow{3}{1cm}{España} & Madrid & Manzanares & Cibeles\\ \cline{2-4}
%& Sevilla & Guadalquivir & Giralda\\ \cline{2-4}
%& Zaragoza & Ebro & Pilar\\ \cline{1-4}
%Francia & París & Sena & Torre Eiffel\\ \cline{1-4}
%\multirow{2}{1cm}{Italia} & Roma & Tíber & San Pedro\\ \cline{2-4}
%& Milán & \multicolumn{1}{c|}{-} & Duomo\\ \cline{1-4}
%\end{tabular}
%\caption{Tabla muy bonita.}
%\label{tabla:final}
%\end{table}
%%%%%%%%%%%%%%%Introducir una tabla que ocupe ambas columnas
%\begin{table*}[htb]
%\centering
%\begin{tabular}{p{0.2\textwidth} p{0.7\textwidth}}
%\hline
%Montaña & Descripción \\
%\hline \hline
%Monte Elbrus & Se encuentra en Rusia, muy cerca de Georgia. Es la montana más alta de Europa. \\
%\hline
%Mont Blanc & Se encuentra en la frontera entre Francia e Italia. Erróneamente, suele decirse que es la más alta de Europa. \\
%\hline
%\end{tabular}
%\caption{Montañas.}
%\label{tabla:montanas}
%\end{table*}
\section{Conclusión}
En 1917 se observó in vitro que extractos de glándula pineal producían un aclaramiento en la piel de sapo. A finales de los 50, Lerner y colaboradores aislaron la hormona pineal a partir de pinealocitos bovinos y describieron su estructura química: 5-metoxi-N-acetiltriptamina (melatonina). Si bien durante mucho tiempo se consideró que la melatonina era de origen exclusivamente cerebral, se ha demostrado la biosíntesis del metoxindol en otros tejidos como la retina, la glándula harderiana, el hígado, el intestino, los riñones, las glandulas suprarrenales, el timo, la glándula tiroides, las células inmunes, el páncreas, los ovarios, el cuerpo carotídeo, la placenta y el endometrio.
En el Homo sapiens se produce una síntesis constante de melatonina que disminuye abruptamente hacia los 30 años de edad, a partir de la adolescencia suelen producirse calcificaciones en la glándula que reciben el nombre de corpora arenacea.1​ Se ha observado que la melatonina tiene, entre otras funciones, regular el reloj biológico y disminuir la oxidación. Los déficits de melatonina pueden ir acompañados de insomnio y depresión y podrían provocar una paulatina aceleración del envejecimiento.
\bibliographystyle{plain}
\bibliography{bibliografia}@Article{•,
author = {•},
title = {•},
journal = {•},
year = {•},
OPTkey = {•},
OPTvolume = {•},
OPTnumber = {•},
OPTpages = {•},
OPTmonth = {•},
OPTnote = {•},
OPTannote = {•}
}

\end{document}
